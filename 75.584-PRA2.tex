% Options for packages loaded elsewhere
\PassOptionsToPackage{unicode}{hyperref}
\PassOptionsToPackage{hyphens}{url}
%
\documentclass[
]{article}
\usepackage{amsmath,amssymb}
\usepackage{lmodern}
\usepackage{iftex}
\ifPDFTeX
  \usepackage[T1]{fontenc}
  \usepackage[utf8]{inputenc}
  \usepackage{textcomp} % provide euro and other symbols
\else % if luatex or xetex
  \usepackage{unicode-math}
  \defaultfontfeatures{Scale=MatchLowercase}
  \defaultfontfeatures[\rmfamily]{Ligatures=TeX,Scale=1}
\fi
% Use upquote if available, for straight quotes in verbatim environments
\IfFileExists{upquote.sty}{\usepackage{upquote}}{}
\IfFileExists{microtype.sty}{% use microtype if available
  \usepackage[]{microtype}
  \UseMicrotypeSet[protrusion]{basicmath} % disable protrusion for tt fonts
}{}
\makeatletter
\@ifundefined{KOMAClassName}{% if non-KOMA class
  \IfFileExists{parskip.sty}{%
    \usepackage{parskip}
  }{% else
    \setlength{\parindent}{0pt}
    \setlength{\parskip}{6pt plus 2pt minus 1pt}}
}{% if KOMA class
  \KOMAoptions{parskip=half}}
\makeatother
\usepackage{xcolor}
\usepackage[margin=1in]{geometry}
\usepackage{color}
\usepackage{fancyvrb}
\newcommand{\VerbBar}{|}
\newcommand{\VERB}{\Verb[commandchars=\\\{\}]}
\DefineVerbatimEnvironment{Highlighting}{Verbatim}{commandchars=\\\{\}}
% Add ',fontsize=\small' for more characters per line
\usepackage{framed}
\definecolor{shadecolor}{RGB}{48,48,48}
\newenvironment{Shaded}{\begin{snugshade}}{\end{snugshade}}
\newcommand{\AlertTok}[1]{\textcolor[rgb]{1.00,0.81,0.69}{#1}}
\newcommand{\AnnotationTok}[1]{\textcolor[rgb]{0.50,0.62,0.50}{\textbf{#1}}}
\newcommand{\AttributeTok}[1]{\textcolor[rgb]{0.80,0.80,0.80}{#1}}
\newcommand{\BaseNTok}[1]{\textcolor[rgb]{0.86,0.64,0.64}{#1}}
\newcommand{\BuiltInTok}[1]{\textcolor[rgb]{0.80,0.80,0.80}{#1}}
\newcommand{\CharTok}[1]{\textcolor[rgb]{0.86,0.64,0.64}{#1}}
\newcommand{\CommentTok}[1]{\textcolor[rgb]{0.50,0.62,0.50}{#1}}
\newcommand{\CommentVarTok}[1]{\textcolor[rgb]{0.50,0.62,0.50}{\textbf{#1}}}
\newcommand{\ConstantTok}[1]{\textcolor[rgb]{0.86,0.64,0.64}{\textbf{#1}}}
\newcommand{\ControlFlowTok}[1]{\textcolor[rgb]{0.94,0.87,0.69}{#1}}
\newcommand{\DataTypeTok}[1]{\textcolor[rgb]{0.87,0.87,0.75}{#1}}
\newcommand{\DecValTok}[1]{\textcolor[rgb]{0.86,0.86,0.80}{#1}}
\newcommand{\DocumentationTok}[1]{\textcolor[rgb]{0.50,0.62,0.50}{#1}}
\newcommand{\ErrorTok}[1]{\textcolor[rgb]{0.76,0.75,0.62}{#1}}
\newcommand{\ExtensionTok}[1]{\textcolor[rgb]{0.80,0.80,0.80}{#1}}
\newcommand{\FloatTok}[1]{\textcolor[rgb]{0.75,0.75,0.82}{#1}}
\newcommand{\FunctionTok}[1]{\textcolor[rgb]{0.94,0.94,0.56}{#1}}
\newcommand{\ImportTok}[1]{\textcolor[rgb]{0.80,0.80,0.80}{#1}}
\newcommand{\InformationTok}[1]{\textcolor[rgb]{0.50,0.62,0.50}{\textbf{#1}}}
\newcommand{\KeywordTok}[1]{\textcolor[rgb]{0.94,0.87,0.69}{#1}}
\newcommand{\NormalTok}[1]{\textcolor[rgb]{0.80,0.80,0.80}{#1}}
\newcommand{\OperatorTok}[1]{\textcolor[rgb]{0.94,0.94,0.82}{#1}}
\newcommand{\OtherTok}[1]{\textcolor[rgb]{0.94,0.94,0.56}{#1}}
\newcommand{\PreprocessorTok}[1]{\textcolor[rgb]{1.00,0.81,0.69}{\textbf{#1}}}
\newcommand{\RegionMarkerTok}[1]{\textcolor[rgb]{0.80,0.80,0.80}{#1}}
\newcommand{\SpecialCharTok}[1]{\textcolor[rgb]{0.86,0.64,0.64}{#1}}
\newcommand{\SpecialStringTok}[1]{\textcolor[rgb]{0.80,0.58,0.58}{#1}}
\newcommand{\StringTok}[1]{\textcolor[rgb]{0.80,0.58,0.58}{#1}}
\newcommand{\VariableTok}[1]{\textcolor[rgb]{0.80,0.80,0.80}{#1}}
\newcommand{\VerbatimStringTok}[1]{\textcolor[rgb]{0.80,0.58,0.58}{#1}}
\newcommand{\WarningTok}[1]{\textcolor[rgb]{0.50,0.62,0.50}{\textbf{#1}}}
\usepackage{graphicx}
\makeatletter
\def\maxwidth{\ifdim\Gin@nat@width>\linewidth\linewidth\else\Gin@nat@width\fi}
\def\maxheight{\ifdim\Gin@nat@height>\textheight\textheight\else\Gin@nat@height\fi}
\makeatother
% Scale images if necessary, so that they will not overflow the page
% margins by default, and it is still possible to overwrite the defaults
% using explicit options in \includegraphics[width, height, ...]{}
\setkeys{Gin}{width=\maxwidth,height=\maxheight,keepaspectratio}
% Set default figure placement to htbp
\makeatletter
\def\fps@figure{htbp}
\makeatother
\setlength{\emergencystretch}{3em} % prevent overfull lines
\providecommand{\tightlist}{%
  \setlength{\itemsep}{0pt}\setlength{\parskip}{0pt}}
\setcounter{secnumdepth}{-\maxdimen} % remove section numbering
\ifLuaTeX
  \usepackage{selnolig}  % disable illegal ligatures
\fi
\IfFileExists{bookmark.sty}{\usepackage{bookmark}}{\usepackage{hyperref}}
\IfFileExists{xurl.sty}{\usepackage{xurl}}{} % add URL line breaks if available
\urlstyle{same} % disable monospaced font for URLs
\hypersetup{
  pdftitle={Práctica 2: ¿Cómo realizar la limpieza y análisis de datos?},
  pdfauthor={Autores: Ouassim Aouattah Akandouch, Juan Andrés Dávila},
  hidelinks,
  pdfcreator={LaTeX via pandoc}}

\title{Práctica 2: ¿Cómo realizar la limpieza y análisis de datos?}
\usepackage{etoolbox}
\makeatletter
\providecommand{\subtitle}[1]{% add subtitle to \maketitle
  \apptocmd{\@title}{\par {\large #1 \par}}{}{}
}
\makeatother
\subtitle{Tipología y ciclo de vida de los datos}
\author{Autores: Ouassim Aouattah Akandouch, Juan Andrés Dávila}
\date{Enero 2023}

\begin{document}
\maketitle

{
\setcounter{tocdepth}{2}
\tableofcontents
}
\begin{center}\rule{0.5\linewidth}{0.5pt}\end{center}

\hypertarget{descripciuxf3n-del-dataset}{%
\section{1. Descripción del dataset}\label{descripciuxf3n-del-dataset}}

\begin{center}\rule{0.5\linewidth}{0.5pt}\end{center}

\hypertarget{descripciuxf3n-del-dataset.-por-quuxe9-es-importante-y-quuxe9-preguntaproblema-pretende-responder}{%
\subsection{Descripción del dataset. ¿Por qué es importante y qué
pregunta/problema pretende
responder?}\label{descripciuxf3n-del-dataset.-por-quuxe9-es-importante-y-quuxe9-preguntaproblema-pretende-responder}}

Este dataset, llamado \emph{Heart Attack Analysis \& Prediction
Dataset}, contiene información sobre pacientes que han sufrido o podrían
sufrir un ataque al corazón. Incluye atributos como la edad y el sexo
del paciente, si ha experimentado angina inducida por ejercicio, el
número de vasos principales, el tipo de dolor en el pecho, la presión
arterial en reposo, el colesterol, los resultados electrocardiográficos
en reposo, la frecuencia cardíaca máxima lograda, y finalmente si tiene
más o menos probabilidades de sufrir un ataque al corazón.

Este dataset es importante ya que el ataque al corazón es una de las
principales causas de mortalidad a nivel mundial. El análisis de estos
datos puede ayudar a identificar patrones y factores de riesgo
relacionados con los ataques cardíacos, lo que permitiría a los médicos
y científicos de la salud tomar medidas preventivas y mejorar los
tratamientos.Se puede responder preguntas como ¿Qué factor o variables
son los de mayor incidencia cuando una persona sufre de un ataque al
corazón?o ¿Las mujeres son mas propensas a tener este tipo de
enfermedades? ¿es el colesterol un factor más determinante para un
ataque cardiaco que la edad de la persona?

Revisamos la descripción de las variables contenidas en el fichero y los
tipos de cada una y a continuación Construimos un pequeño diccionario de
datos utilizando la documentación auxiliar.

\begin{itemize}
\tightlist
\item
  \texttt{target}: Probabilidades de sufrir un ataque al corazón (0 =
  menor probabilidad, 1 = mayor probabilidad).
\item
  \texttt{Age}: Edad del paciente (en años).
\item
  \texttt{trtbps}: Presión arterial en reposo (en mm Hg).
\item
  \texttt{cp}: Tipo de dolor en el pecho.

  \begin{itemize}
  \tightlist
  \item
    Valor 1: Angina típica.
  \item
    Valor 2: Angina atípica.
  \item
    Valor 3: Dolor no anginal.
  \item
    Valor 4: Asintomático.
  \end{itemize}
\item
  \texttt{thalach}: Frecuencia cardíaca máxima alcanzada (en latidos por
  minuto).
\item
  \texttt{Sex}: Sexo del paciente (1 = hombre, 0 = mujer).
\item
  \texttt{exang}: Angina inducida por ejercicio (1 = sí, 0 = no).
\item
  \texttt{caa}: Número de vasos principales (valores posibles: 0-3)
\item
  \texttt{chol}: Colesterol (en mg/dl) obtenido a través del sensor de
  índice de masa corporal (IMC).
\item
  \texttt{fbs}: Azúcar en sangre en ayuno \textgreater{} 120 mg/dl (1 =
  verdadero, 0 = falso).
\item
  \texttt{rest\_ecg}: Resultados electrocardiográficos en reposo.

  \begin{itemize}
  \tightlist
  \item
    Valor 0: Normal.
  \item
    Valor 1: Anormalidades en las ondas ST-T (inversiones de onda T y/o
    elevación o depresión ST \textgreater{} 0.05 mV).
  \item
    Valor 2: Hipertrofia ventricular izquierda probable o definitiva
    según los criterios de Estes.
  \end{itemize}
\end{itemize}

\hypertarget{integraciuxf3n-y-selecciuxf3n}{%
\section{2. Integración y
selección}\label{integraciuxf3n-y-selecciuxf3n}}

\hypertarget{integraciuxf3n-y-selecciuxf3n-de-los-datos-de-interuxe9s-a-analizar.-puede-ser-el-resultado-de-adicionar-diferentes-datasets-o-una-subselecciuxf3n-uxfatil-de-los-datos-originales-en-base-al-objetivo-que-se-quiera-conseguir.}{%
\subsection{Integración y selección de los datos de interés a analizar.
Puede ser el resultado de adicionar diferentes datasets o una
subselección útil de los datos originales, en base al objetivo que se
quiera
conseguir.}\label{integraciuxf3n-y-selecciuxf3n-de-los-datos-de-interuxe9s-a-analizar.-puede-ser-el-resultado-de-adicionar-diferentes-datasets-o-una-subselecciuxf3n-uxfatil-de-los-datos-originales-en-base-al-objetivo-que-se-quiera-conseguir.}}

Como se indicó en el apartado anterior existen muchas variables con las
cuales se pueden trabajar para poder realizar un análisis por lo que
empezaremos cargando el fichero de datos.

\begin{Shaded}
\begin{Highlighting}[]
\NormalTok{path }\OtherTok{=} \StringTok{\textquotesingle{}heart.csv\textquotesingle{}}
\NormalTok{df }\OtherTok{\textless{}{-}} \FunctionTok{read.csv}\NormalTok{(path, }\AttributeTok{row.names=}\ConstantTok{NULL}\NormalTok{)}
\end{Highlighting}
\end{Shaded}

Verificamos la estructura del juego de datos principal. Mostramos el
número de columnas que tenemos y ejemplos de los contenidos de las
filas.

\begin{Shaded}
\begin{Highlighting}[]
\NormalTok{structure }\OtherTok{=} \FunctionTok{str}\NormalTok{(df)}
\end{Highlighting}
\end{Shaded}

\begin{verbatim}
## 'data.frame':    303 obs. of  14 variables:
##  $ age     : int  63 37 41 56 57 57 56 44 52 57 ...
##  $ sex     : int  1 1 0 1 0 1 0 1 1 1 ...
##  $ cp      : int  3 2 1 1 0 0 1 1 2 2 ...
##  $ trtbps  : int  145 130 130 120 120 140 140 120 172 150 ...
##  $ chol    : int  233 250 204 236 354 192 294 263 199 168 ...
##  $ fbs     : int  1 0 0 0 0 0 0 0 1 0 ...
##  $ restecg : int  0 1 0 1 1 1 0 1 1 1 ...
##  $ thalachh: int  150 187 172 178 163 148 153 173 162 174 ...
##  $ exng    : int  0 0 0 0 1 0 0 0 0 0 ...
##  $ oldpeak : num  2.3 3.5 1.4 0.8 0.6 0.4 1.3 0 0.5 1.6 ...
##  $ slp     : int  0 0 2 2 2 1 1 2 2 2 ...
##  $ caa     : int  0 0 0 0 0 0 0 0 0 0 ...
##  $ thall   : int  1 2 2 2 2 1 2 3 3 2 ...
##  $ output  : int  1 1 1 1 1 1 1 1 1 1 ...
\end{verbatim}

\begin{Shaded}
\begin{Highlighting}[]
\FunctionTok{summary}\NormalTok{(df)}
\end{Highlighting}
\end{Shaded}

\begin{verbatim}
##       age             sex               cp            trtbps     
##  Min.   :29.00   Min.   :0.0000   Min.   :0.000   Min.   : 94.0  
##  1st Qu.:47.50   1st Qu.:0.0000   1st Qu.:0.000   1st Qu.:120.0  
##  Median :55.00   Median :1.0000   Median :1.000   Median :130.0  
##  Mean   :54.37   Mean   :0.6832   Mean   :0.967   Mean   :131.6  
##  3rd Qu.:61.00   3rd Qu.:1.0000   3rd Qu.:2.000   3rd Qu.:140.0  
##  Max.   :77.00   Max.   :1.0000   Max.   :3.000   Max.   :200.0  
##       chol            fbs            restecg          thalachh    
##  Min.   :126.0   Min.   :0.0000   Min.   :0.0000   Min.   : 71.0  
##  1st Qu.:211.0   1st Qu.:0.0000   1st Qu.:0.0000   1st Qu.:133.5  
##  Median :240.0   Median :0.0000   Median :1.0000   Median :153.0  
##  Mean   :246.3   Mean   :0.1485   Mean   :0.5281   Mean   :149.6  
##  3rd Qu.:274.5   3rd Qu.:0.0000   3rd Qu.:1.0000   3rd Qu.:166.0  
##  Max.   :564.0   Max.   :1.0000   Max.   :2.0000   Max.   :202.0  
##       exng           oldpeak          slp             caa        
##  Min.   :0.0000   Min.   :0.00   Min.   :0.000   Min.   :0.0000  
##  1st Qu.:0.0000   1st Qu.:0.00   1st Qu.:1.000   1st Qu.:0.0000  
##  Median :0.0000   Median :0.80   Median :1.000   Median :0.0000  
##  Mean   :0.3267   Mean   :1.04   Mean   :1.399   Mean   :0.7294  
##  3rd Qu.:1.0000   3rd Qu.:1.60   3rd Qu.:2.000   3rd Qu.:1.0000  
##  Max.   :1.0000   Max.   :6.20   Max.   :2.000   Max.   :4.0000  
##      thall           output      
##  Min.   :0.000   Min.   :0.0000  
##  1st Qu.:2.000   1st Qu.:0.0000  
##  Median :2.000   Median :1.0000  
##  Mean   :2.314   Mean   :0.5446  
##  3rd Qu.:3.000   3rd Qu.:1.0000  
##  Max.   :3.000   Max.   :1.0000
\end{verbatim}

Vemos que tenemos \textbf{14} variables y \textbf{303} registros.

En este caso podemos deshacernos de variables que no no seran utilizadas
para el análisis de esta practica como el azucar en sangre o el numero
de vasos princiaples.

\begin{Shaded}
\begin{Highlighting}[]
\NormalTok{df }\OtherTok{\textless{}{-}}\NormalTok{ df[,}\SpecialCharTok{!}\FunctionTok{names}\NormalTok{(df) }\SpecialCharTok{\%in\%} \FunctionTok{c}\NormalTok{(}\StringTok{"caa"}\NormalTok{, }\StringTok{"fbs"}\NormalTok{, }\StringTok{"rest\_ecg"}\NormalTok{)]}
\FunctionTok{summary}\NormalTok{(df)}
\end{Highlighting}
\end{Shaded}

\begin{verbatim}
##       age             sex               cp            trtbps     
##  Min.   :29.00   Min.   :0.0000   Min.   :0.000   Min.   : 94.0  
##  1st Qu.:47.50   1st Qu.:0.0000   1st Qu.:0.000   1st Qu.:120.0  
##  Median :55.00   Median :1.0000   Median :1.000   Median :130.0  
##  Mean   :54.37   Mean   :0.6832   Mean   :0.967   Mean   :131.6  
##  3rd Qu.:61.00   3rd Qu.:1.0000   3rd Qu.:2.000   3rd Qu.:140.0  
##  Max.   :77.00   Max.   :1.0000   Max.   :3.000   Max.   :200.0  
##       chol          restecg          thalachh          exng       
##  Min.   :126.0   Min.   :0.0000   Min.   : 71.0   Min.   :0.0000  
##  1st Qu.:211.0   1st Qu.:0.0000   1st Qu.:133.5   1st Qu.:0.0000  
##  Median :240.0   Median :1.0000   Median :153.0   Median :0.0000  
##  Mean   :246.3   Mean   :0.5281   Mean   :149.6   Mean   :0.3267  
##  3rd Qu.:274.5   3rd Qu.:1.0000   3rd Qu.:166.0   3rd Qu.:1.0000  
##  Max.   :564.0   Max.   :2.0000   Max.   :202.0   Max.   :1.0000  
##     oldpeak          slp            thall           output      
##  Min.   :0.00   Min.   :0.000   Min.   :0.000   Min.   :0.0000  
##  1st Qu.:0.00   1st Qu.:1.000   1st Qu.:2.000   1st Qu.:0.0000  
##  Median :0.80   Median :1.000   Median :2.000   Median :1.0000  
##  Mean   :1.04   Mean   :1.399   Mean   :2.314   Mean   :0.5446  
##  3rd Qu.:1.60   3rd Qu.:2.000   3rd Qu.:3.000   3rd Qu.:1.0000  
##  Max.   :6.20   Max.   :2.000   Max.   :3.000   Max.   :1.0000
\end{verbatim}

\hypertarget{limpieza-de-los-datos.}{%
\section{3. Limpieza de los datos.}\label{limpieza-de-los-datos.}}

\hypertarget{los-datos-contienen-ceros-o-elementos-vacuxedos-gestiona-cada-uno-de-estos-casos.}{%
\subsection{3.1. ¿Los datos contienen ceros o elementos vacíos? Gestiona
cada uno de estos
casos.}\label{los-datos-contienen-ceros-o-elementos-vacuxedos-gestiona-cada-uno-de-estos-casos.}}

Un paso esencial será la limpieza de datos, mirando si hay valores
vacíos o nulos.

\begin{Shaded}
\begin{Highlighting}[]
\FunctionTok{colSums}\NormalTok{(}\FunctionTok{is.na}\NormalTok{(df))}
\end{Highlighting}
\end{Shaded}

\begin{verbatim}
##      age      sex       cp   trtbps     chol  restecg thalachh     exng 
##        0        0        0        0        0        0        0        0 
##  oldpeak      slp    thall   output 
##        0        0        0        0
\end{verbatim}

\begin{Shaded}
\begin{Highlighting}[]
\FunctionTok{colSums}\NormalTok{(df }\SpecialCharTok{==} \StringTok{""}\NormalTok{)}
\end{Highlighting}
\end{Shaded}

\begin{verbatim}
##      age      sex       cp   trtbps     chol  restecg thalachh     exng 
##        0        0        0        0        0        0        0        0 
##  oldpeak      slp    thall   output 
##        0        0        0        0
\end{verbatim}

Vemos que no hay valores nulos en los datos ni existen campos llenos de
espacios en blanco. Procedemos a convertir a factores las variables
correspondientes a este tipo.

\begin{Shaded}
\begin{Highlighting}[]
\NormalTok{df}\SpecialCharTok{$}\NormalTok{exng }\OtherTok{\textless{}{-}} \FunctionTok{factor}\NormalTok{(df}\SpecialCharTok{$}\NormalTok{exng)}
\NormalTok{df}\SpecialCharTok{$}\NormalTok{cp }\OtherTok{\textless{}{-}} \FunctionTok{factor}\NormalTok{(df}\SpecialCharTok{$}\NormalTok{cp)}
\NormalTok{df}\SpecialCharTok{$}\NormalTok{output }\OtherTok{\textless{}{-}} \FunctionTok{factor}\NormalTok{(df}\SpecialCharTok{$}\NormalTok{output)}
\NormalTok{structure }\OtherTok{=} \FunctionTok{str}\NormalTok{(df)}
\end{Highlighting}
\end{Shaded}

\begin{verbatim}
## 'data.frame':    303 obs. of  12 variables:
##  $ age     : int  63 37 41 56 57 57 56 44 52 57 ...
##  $ sex     : int  1 1 0 1 0 1 0 1 1 1 ...
##  $ cp      : Factor w/ 4 levels "0","1","2","3": 4 3 2 2 1 1 2 2 3 3 ...
##  $ trtbps  : int  145 130 130 120 120 140 140 120 172 150 ...
##  $ chol    : int  233 250 204 236 354 192 294 263 199 168 ...
##  $ restecg : int  0 1 0 1 1 1 0 1 1 1 ...
##  $ thalachh: int  150 187 172 178 163 148 153 173 162 174 ...
##  $ exng    : Factor w/ 2 levels "0","1": 1 1 1 1 2 1 1 1 1 1 ...
##  $ oldpeak : num  2.3 3.5 1.4 0.8 0.6 0.4 1.3 0 0.5 1.6 ...
##  $ slp     : int  0 0 2 2 2 1 1 2 2 2 ...
##  $ thall   : int  1 2 2 2 2 1 2 3 3 2 ...
##  $ output  : Factor w/ 2 levels "0","1": 2 2 2 2 2 2 2 2 2 2 ...
\end{verbatim}

\hypertarget{identifica-y-gestiona-los-valores-extremos}{%
\subsection{3.2. Identifica y gestiona los valores
extremos}\label{identifica-y-gestiona-los-valores-extremos}}

A continuación, analizaremos los posibles valores extremos de nuestro
juego de datos.

\begin{Shaded}
\begin{Highlighting}[]
\FunctionTok{par}\NormalTok{(}\AttributeTok{mfrow =} \FunctionTok{c}\NormalTok{(}\DecValTok{3}\NormalTok{,}\DecValTok{2}\NormalTok{))}
\ControlFlowTok{for}\NormalTok{(i }\ControlFlowTok{in} \DecValTok{1}\SpecialCharTok{:}\FunctionTok{ncol}\NormalTok{(df)) \{}
 \FunctionTok{boxplot}\NormalTok{(df[, i], }\AttributeTok{main =} \FunctionTok{colnames}\NormalTok{(df)[i],}\AttributeTok{col=}\StringTok{"blue"}\NormalTok{)}
  \FunctionTok{plot}\NormalTok{(df[, i])}
\NormalTok{\}}
\end{Highlighting}
\end{Shaded}

\includegraphics{75.584-PRA2_files/figure-latex/unnamed-chunk-6-1.pdf}
\includegraphics{75.584-PRA2_files/figure-latex/unnamed-chunk-6-2.pdf}
\includegraphics{75.584-PRA2_files/figure-latex/unnamed-chunk-6-3.pdf}
\includegraphics{75.584-PRA2_files/figure-latex/unnamed-chunk-6-4.pdf}

\hypertarget{anuxe1lisis-de-los-datos.}{%
\section{4. Análisis de los datos.}\label{anuxe1lisis-de-los-datos.}}

\hypertarget{selecciuxf3n-de-los-grupos-de-datos-que-se-quieren-analizarcomparar-p.-ej.-si-se-van-a-comparar-grupos-de-datos-cuuxe1les-son-estos-grupos-y-quuxe9-tipo-de-anuxe1lisis-se-van-a-aplicar}{%
\subsection{4.1. Selección de los grupos de datos que se quieren
analizar/comparar (p.~ej., si se van a comparar grupos de datos, ¿cuáles
son estos grupos y qué tipo de análisis se van a
aplicar?)}\label{selecciuxf3n-de-los-grupos-de-datos-que-se-quieren-analizarcomparar-p.-ej.-si-se-van-a-comparar-grupos-de-datos-cuuxe1les-son-estos-grupos-y-quuxe9-tipo-de-anuxe1lisis-se-van-a-aplicar}}

Como primer punto a analizar vamos a verificar si el nivel de colesterol
medido en miligramos por decilitro mg/dl tiene una relacion más directa
que la edad con la probabilidad de sufrir un ataque cardiaco que la Para
ellos creamos un nuevo dataset con estas dos variables.

\begin{Shaded}
\begin{Highlighting}[]
\NormalTok{test }\OtherTok{\textless{}{-}}\NormalTok{ df[,}\FunctionTok{names}\NormalTok{(df) }\SpecialCharTok{\%in\%} \FunctionTok{c}\NormalTok{(}\StringTok{"chol"}\NormalTok{, }\StringTok{"output"}\NormalTok{,}\StringTok{"age"}\NormalTok{)]}
\NormalTok{structure}\OtherTok{=} \FunctionTok{str}\NormalTok{(test)}
\end{Highlighting}
\end{Shaded}

\begin{verbatim}
## 'data.frame':    303 obs. of  3 variables:
##  $ age   : int  63 37 41 56 57 57 56 44 52 57 ...
##  $ chol  : int  233 250 204 236 354 192 294 263 199 168 ...
##  $ output: Factor w/ 2 levels "0","1": 2 2 2 2 2 2 2 2 2 2 ...
\end{verbatim}

\hypertarget{comprobaciuxf3n-de-la-normalidad-y-homogeneidad-de-la-varianza.}{%
\subsection{4.2. Comprobación de la normalidad y homogeneidad de la
varianza.}\label{comprobaciuxf3n-de-la-normalidad-y-homogeneidad-de-la-varianza.}}

En cuanto al primer dataset creado vamos a comprobar la normalidad la
variable col

\begin{Shaded}
\begin{Highlighting}[]
\FunctionTok{grid.arrange}\NormalTok{((}\FunctionTok{ggdensity}\NormalTok{(test}\SpecialCharTok{$}\NormalTok{chol, }\AttributeTok{fill =} \StringTok{"lightgreen"}\NormalTok{)),}\AttributeTok{ncol=}\DecValTok{1}\NormalTok{)}
\end{Highlighting}
\end{Shaded}

\includegraphics{75.584-PRA2_files/figure-latex/unnamed-chunk-8-1.pdf}

\begin{Shaded}
\begin{Highlighting}[]
\FunctionTok{grid.arrange}\NormalTok{((}\FunctionTok{ggdensity}\NormalTok{(test}\SpecialCharTok{$}\NormalTok{age, }\AttributeTok{fill =} \StringTok{"lightgreen"}\NormalTok{)),}\AttributeTok{ncol=}\DecValTok{1}\NormalTok{)}
\end{Highlighting}
\end{Shaded}

\includegraphics{75.584-PRA2_files/figure-latex/unnamed-chunk-8-2.pdf} A
través de un gráfico de densidad podemos ver la distribución de valores
para la variable peso y demostrar que toma la forma de campana esperada
que indica normalidad.

\hypertarget{aplicaciuxf3n-de-pruebas-estaduxedsticas-para-comparar-los-grupos-de-datos.en-funciuxf3n-de-los-datos-y-el-objetivo-del-estudio-aplicar-pruebas-de-contraste-de-hipuxf3tesis-correlaciones-regresiones-etc.-aplicar-al-menos-tres-muxe9todos-de-anuxe1lisis-diferentes.}{%
\subsection{4.3. Aplicación de pruebas estadísticas para comparar los
grupos de datos.En función de los datos y el objetivo del estudio,
aplicar pruebas de contraste de hipótesis, correlaciones, regresiones,
etc. Aplicar al menos tres métodos de análisis
diferentes.}\label{aplicaciuxf3n-de-pruebas-estaduxedsticas-para-comparar-los-grupos-de-datos.en-funciuxf3n-de-los-datos-y-el-objetivo-del-estudio-aplicar-pruebas-de-contraste-de-hipuxf3tesis-correlaciones-regresiones-etc.-aplicar-al-menos-tres-muxe9todos-de-anuxe1lisis-diferentes.}}

Utilizamos un modelo de regresión logistica en este caso

\begin{Shaded}
\begin{Highlighting}[]
\NormalTok{modeloRL }\OtherTok{\textless{}{-}} \FunctionTok{glm}\NormalTok{(output }\SpecialCharTok{\textasciitilde{}}\NormalTok{ chol, }\AttributeTok{data =}\NormalTok{ test, }\AttributeTok{family =} \StringTok{"binomial"}\NormalTok{)}
\FunctionTok{summary}\NormalTok{(modeloRL)}
\end{Highlighting}
\end{Shaded}

\begin{verbatim}
## 
## Call:
## glm(formula = output ~ chol, family = "binomial", data = test)
## 
## Deviance Residuals: 
##    Min      1Q  Median      3Q     Max  
## -1.425  -1.241   1.015   1.093   1.567  
## 
## Coefficients:
##              Estimate Std. Error z value Pr(>|z|)  
## (Intercept)  1.001617   0.571467   1.753   0.0797 .
## chol        -0.003338   0.002269  -1.471   0.1412  
## ---
## Signif. codes:  0 '***' 0.001 '**' 0.01 '*' 0.05 '.' 0.1 ' ' 1
## 
## (Dispersion parameter for binomial family taken to be 1)
## 
##     Null deviance: 417.64  on 302  degrees of freedom
## Residual deviance: 415.43  on 301  degrees of freedom
## AIC: 419.43
## 
## Number of Fisher Scoring iterations: 4
\end{verbatim}

\begin{Shaded}
\begin{Highlighting}[]
\FunctionTok{plot}\NormalTok{(output }\SpecialCharTok{\textasciitilde{}}\NormalTok{ chol, test, }\AttributeTok{col =} \StringTok{"darkblue"}\NormalTok{,}
     \AttributeTok{main =} \StringTok{"Modelo regresión logística"}\NormalTok{,}
     \AttributeTok{ylab =} \StringTok{"P(output=1|colesterol)"}\NormalTok{,}
     \AttributeTok{xlab =} \StringTok{"colesterol"}\NormalTok{, }\AttributeTok{pch =} \StringTok{"I"}\NormalTok{)}

\CommentTok{\# type = "response" devuelve las predicciones en forma de probabilidad en lugar de en log\_ODDs}
\FunctionTok{curve}\NormalTok{(}\FunctionTok{predict}\NormalTok{(modeloRL, }\FunctionTok{data.frame}\NormalTok{(}\AttributeTok{chol =}\NormalTok{ x), }\AttributeTok{type =} \StringTok{"response"}\NormalTok{),}
      \AttributeTok{col =} \StringTok{"firebrick"}\NormalTok{, }\AttributeTok{lwd =} \FloatTok{2.5}\NormalTok{, }\AttributeTok{add =} \ConstantTok{TRUE}\NormalTok{)}
\end{Highlighting}
\end{Shaded}

\includegraphics{75.584-PRA2_files/figure-latex/unnamed-chunk-10-1.pdf}

\begin{Shaded}
\begin{Highlighting}[]
\NormalTok{modeloRL }\OtherTok{\textless{}{-}} \FunctionTok{glm}\NormalTok{(output }\SpecialCharTok{\textasciitilde{}}\NormalTok{ age, }\AttributeTok{data =}\NormalTok{ test, }\AttributeTok{family =} \StringTok{"binomial"}\NormalTok{)}
\FunctionTok{summary}\NormalTok{(modeloRL)}
\end{Highlighting}
\end{Shaded}

\begin{verbatim}
## 
## Call:
## glm(formula = output ~ age, family = "binomial", data = test)
## 
## Deviance Residuals: 
##     Min       1Q   Median       3Q      Max  
## -1.7125  -1.1773   0.8296   1.0685   1.5947  
## 
## Coefficients:
##             Estimate Std. Error z value Pr(>|z|)    
## (Intercept)  3.03623    0.75639   4.014 5.97e-05 ***
## age         -0.05235    0.01363  -3.841 0.000122 ***
## ---
## Signif. codes:  0 '***' 0.001 '**' 0.01 '*' 0.05 '.' 0.1 ' ' 1
## 
## (Dispersion parameter for binomial family taken to be 1)
## 
##     Null deviance: 417.64  on 302  degrees of freedom
## Residual deviance: 401.86  on 301  degrees of freedom
## AIC: 405.86
## 
## Number of Fisher Scoring iterations: 4
\end{verbatim}

\begin{Shaded}
\begin{Highlighting}[]
\FunctionTok{plot}\NormalTok{(output }\SpecialCharTok{\textasciitilde{}}\NormalTok{ age, test, }\AttributeTok{col =} \StringTok{"darkblue"}\NormalTok{,}
     \AttributeTok{main =} \StringTok{"Modelo regresión logística"}\NormalTok{,}
     \AttributeTok{ylab =} \StringTok{"P(output=1|colesterol)"}\NormalTok{,}
     \AttributeTok{xlab =} \StringTok{"age"}\NormalTok{, }\AttributeTok{pch =} \StringTok{"I"}\NormalTok{)}

\CommentTok{\# type = "response" devuelve las predicciones en forma de probabilidad en lugar de en log\_ODDs}
\FunctionTok{curve}\NormalTok{(}\FunctionTok{predict}\NormalTok{(modeloRL, }\FunctionTok{data.frame}\NormalTok{(}\AttributeTok{age =}\NormalTok{ x), }\AttributeTok{type =} \StringTok{"response"}\NormalTok{),}
      \AttributeTok{col =} \StringTok{"firebrick"}\NormalTok{, }\AttributeTok{lwd =} \FloatTok{2.5}\NormalTok{, }\AttributeTok{add =} \ConstantTok{TRUE}\NormalTok{)}
\end{Highlighting}
\end{Shaded}

\includegraphics{75.584-PRA2_files/figure-latex/unnamed-chunk-10-2.pdf}

\hypertarget{representaciuxf3n-de-los-resultados-a-partir-de-tablas-y-gruxe1ficas.}{%
\section{5. Representación de los resultados a partir de tablas y
gráficas.}\label{representaciuxf3n-de-los-resultados-a-partir-de-tablas-y-gruxe1ficas.}}

Los resultados graficos se representan en los apartados anteriores.

\hypertarget{resoluciuxf3n-del-problema.-a-partir-de-los-resultados-obtenidos}{%
\section{6. Resolución del problema. A partir de los resultados
obtenidos,}\label{resoluciuxf3n-del-problema.-a-partir-de-los-resultados-obtenidos}}

\hypertarget{conclusiones}{%
\section{Conclusiones}\label{conclusiones}}

Este dataset contiene información sobre pacientes que han tenido o
podrían tener ataques al corazón. Los atributos incluyen la edad y el
sexo del paciente, si el paciente ha experimentado angina inducida por
ejercicio (exang), el número de vasos principales (ca), el tipo de dolor
en el pecho (cp), la presión arterial en reposo (trtbps), el colesterol
(chol), si el paciente tiene niveles elevados de azúcar en sangre en
ayuno (fbs), los resultados electrocardiográficos en reposo (rest\_ecg),
la frecuencia cardíaca máxima lograda (thalach), y si el paciente tiene
más o menos probabilidades de sufrir un ataque al corazón (target). Los
valores para los atributos incluyen información como la edad y el sexo
del paciente, y variables categóricas como el tipo de dolor en el pecho,
los resultados electrocardiográficos y si el paciente tiene más o menos
probabilidades de sufrir un ataque al corazón.

Revisados los datos parecen bien informados. Los datos están limpios y
bien documentados. No plantean graves problemas de campos con valores
nulos o vacíos y tienen bastante potencial para generar nuevos
indicadores a partir de los datos.

Pretendiamos analizar patrones y factores de riesgo relacionados con los
ataques cardíacos, y para desarrollar modelos de predicción para
determinar si un paciente está en riesgo de sufrir un ataque al corazón.

\hypertarget{bibliografuxeda}{%
\section{Bibliografía}\label{bibliografuxeda}}

\begin{itemize}
\tightlist
\item
  Calvo M, Subirats L, Pérez D (2019). \emph{Introducción a la limpieza
  y análisis de los datos}. Editorial UOC.
\item
  Julià Minguillón Alfonso, Ramon Caihuelas Quiles. (2021).
  \emph{Proceso de minería de datos}. Editorial FUOC.
\item
  RASHIK RAHMAN (2021). \emph{Heart Attack Analysis \& Prediction
  Dataset}.
  \url{https://www.kaggle.com/datasets/rashikrahmanpritom/heart-attack-analysis-prediction-dataset?resource=download\&select=heart.csv}
\end{itemize}

\end{document}
